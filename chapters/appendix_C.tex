\chapter{$\vert$ \LaTeX~examples}
\label{chap:appc}

The following environments and tools were used to create this document:
\begin{itemize}
\item operating system: Mac OS X 10.14
\item tex distribution: MacTeX-2014, \url{http://www.tug.org/mactex/}
\item tex editor: Texmaker 5.0.2 for Mac, \url{http://www.xm1math.net/texmaker/} for its XeLaTeX flow (recommended) or pdfLaTeX flow
\item bibtex editor: BibDesk 1.6.3 for Mac, \url{http://bibdesk.sourceforge.net/}
\item fonts \texttt{cslthse-msc.cls} document class): 
\begin{description}
\item{for XeLaTeX}: TeX Gyre Termes, \textsf{TeX Gyre Heros}, \texttt{TeX Gyre Cursor} (installed from the TeXLive 2013)
\item{for pdfLaTeX}: TeX Gyre font packages: tgtermes.sty, tgheros.sty, tgcursor.sty, gtxmath.sty (available through TeXLive 2013) 
\end{description} 
\item picture editor: OmniGraffle Professional 5.4.2
\end{itemize}

\noindent A list of the essential \LaTeX packages needed to compile this document follows (all except \texttt{hyperref} are included in the document class):
\begin{itemize}
\item \texttt{fontspec}, to access local fonts, needs the XeLaTeX flow
\item \texttt{geometry}, for page layout
\item \texttt{titling}, for formatting the title page
\item \texttt{fancyhdr}, for custom headers and footers
\item \texttt{abstract}, for customizing the abstract
\item \texttt{titlesec}, for custom chapters, sections, etc.
\item \texttt{caption}, for custom tables and figure captions
\item \texttt{hyperref}, for producing PDF with hyperlinks
\item \texttt{appendix}, for appendices
\item \texttt{printlen}, for printing text sizes
\item \texttt{textcomp}, for text companion fonts (e.g. bullet)
\item \texttt{pdfpages}, to include the popular science summary page at the end
\end{itemize}

\noindent Other useful packages:
\begin{itemize}
\item \texttt{listings}, for producing code listings with syntax colouring and line numbers
\end{itemize}


\begin{highlightedParagraphC}
 \begin{remark}
     Real-time updating of the receive centrality is approximately a factor $N$ simpler than real-time updating of broadcast centrality for sparse networks.
 \end{remark}
\end{highlightedParagraphC}

\begin{table}[ht]
            \bigskip
		\centering % used for centering table
		\begin{tabular}{c c c c c} % centered columns (4 columns)
			\hline\hline %inserts double horizontal lines
			\textbf{Parameter} & \textbf{\#1} & \textbf{\#2} & \textbf{\#3} & \textbf{\#4} \\ [0.1ex] % inserts table
			%heading
			\hline\hline 
			$\mathbf{\alpha}$ & 0.7 & 0.7 & 0.1 & -\\ % inserting body of the table
			$\mathbf{\beta}$ & 0.1 & 0.01 & 0.1 & - \\ [0.5ex] % [1ex] adds vertical space
			\hline %inserts single line
		\end{tabular}
		\caption{First synthetic experiment: Choice of the $\alpha,\beta$ parameters for the different experimental runs.} 
       \label{table:parameters} 
\end{table}

\newpage

\begin{figure}[htp!]
     \centering
     \begin{subfigure}[b]{0.49\textwidth}
         \centering
         \includegraphics[width=\textwidth]{exp1_bt20a}
         \caption{Broadcast for each of the 31 nodes, $\alpha = 0.7 ,~\beta = 0.1$}
         \label{fig:bt1}
     \end{subfigure}
     %\hspace{0.1cm}
     \begin{subfigure}[b]{0.49\textwidth}
         \centering
         \includegraphics[width=\textwidth]{exp1_bt20b}
         \caption{Broadcast for each of the 31 nodes, $\alpha = 0.7 ,~\beta = 0.01$}
         \label{fig:bt2}
     \end{subfigure}
     
     \begin{subfigure}[b]{0.49\textwidth}
         \centering
         \includegraphics[width=\textwidth]{exp1_bt20c}
         \caption{Broadcast for each of the 31 nodes, $\alpha = 0.1 ,~\beta = 0.1$}
         \label{fig:bt3}
     \end{subfigure}
     %\hspace{0.1cm}
     \begin{subfigure}[b]{0.49\textwidth}
         \centering
         \includegraphics[width=\textwidth]{exp1_agg_out_degree}
         \caption{Aggregate out degree for each node}
         \label{fig:bt4}
     \end{subfigure}
        \caption{First synthetic experiment: results from the dynamic network in Fig. \ref{fig:exp1}.}
        \label{fig:fourbt}
\end{figure}

\begin{figure}[h]\centering
    \includegraphics[width=.65\textwidth]{experiment2}
    \caption{Network structure for the second synthetic experiment. Links of $\mathbf{A}(t)$ are active over non-overlapping time intervals such that $t_i\coloneqq[(i − 1)\tau , (i − 1 + 0.9)\tau )$, for $i=0, 1, \dots , 7$, and $\tau =0.1$, repeated periodically over five cycles.}
    \label{fig:exp2}
    \bigskip
\end{figure}

\begin{figure}
     \centering
     \begin{subfigure}[b]{0.49\textwidth}
         \centering
         \includegraphics[width=\textwidth]{exp2b_btA_vs_btB}
         \caption{$\mathbf{b}(t)$ for $\alpha = 0.7 ,~\beta = 0.1$}
         \label{fig:bt5}
     \end{subfigure}
     \hfill
     \begin{subfigure}[b]{0.49\textwidth}
         \centering
         \includegraphics[width=\textwidth]{exp2a_btA_vs_btB}
         \caption{$\mathbf{b}(t)$ for $\alpha = 0.9 ,~\beta = 0.1$}
         \label{fig:bt6}
     \end{subfigure}
     \caption{Second synthetic experiment: dynamic broadcast centrality over time for node A (solid) and node B (dashed) in the network of Fig. \ref{fig:exp2}.}
     \label{fig:twobt}
\end{figure}

{\small \begin{center}  
        \begin{minipage}{0.95\textwidth}
            \begin{algorithm}[H]
                \SetAlgoLined
                \textbf{Step 1. Start:} Choose $x_0$ and $m+p=$ Krylov dim. Initialize $\widetilde{H}_{m+p} =\textbf{0}$\;
                \textbf{Step 2. Arnoldi iteration:} $r_0=b-Ax_0,~\beta = \|r_0\|_2,~ v_1 = r_0/\beta$\;
                \For{$j = 1,2,\dots,m+p$}{
                    \textbf{if~}$j\le m$ \textbf{ then } $z_j \coloneqq M_j^{-1}v_j$ \textbf{ else } $z_j \coloneqq M_j^{-1}w_{j-m}$\;
                    $w\coloneqq Az_j$\;
                    \For{$i=1,2,\dots, j$}{$h_{i,j}\coloneqq(w,v_i),~~w\coloneqq w -h_{i,j}v_i$}
                    $h_{j+1,j}=\|w\|_2, ~~v_{j+1}=w/h_{j+1,j}$\;
                }
                $Z_{m+p}\coloneqq[z_1,\dots,z_{m+p}],~~\widetilde{H}_{m+p} = [h_{i,j}]_{1\le i \le j+1, 1\le j \le m + p}$\; 
                \textbf{Step 3. Form approx. sol.:} Compute $y$ s.t. min. $\|\beta e_1 - \widetilde{H}_{m+p}y\|_2$\;
                $x_m = x_0 + Z_{m+p}y$\;
                \textbf{Step 4. Restart:} If satisfied Stop, else set $x_0\leftarrow x_m$ and goto Step 2.
            \caption{FGMRES: GMRES with var. preconditioning}
            \end{algorithm}
        \end{minipage}
    \end{center}
    }

\begin{table}[!htpb]
    \centering
    \begin{tabular}{llc}
        \toprule
        \textbf{Header 01} & \textbf{Header 02} & \textbf{Header 03} \\ 
        \midrule
        Lorem Ipsum         & Pharetra Dolor    & $\checkmark$  \\
        Amet Consectetuer   & Curabitur Aliquet & -             \\
        Praesent Mauris     & Praesent Libero   & $\checkmark$  \\
        \bottomrule
    \end{tabular}
    \caption{A table showcasing the usage of the tabular environment.}
    \label{tab:table-01}
\end{table}